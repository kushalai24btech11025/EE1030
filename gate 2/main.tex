\let\negmedspace\undefined
\let\negthickspace\undefined
\documentclass[journal]{IEEEtran}
\usepackage[a5paper, margin=10mm, onecolumn]{geometry}
%\usepackage{lmodern} % Ensure lmodern is loaded for pdflatex
\usepackage{tfrupee} % Include tfrupee package

\setlength{\headheight}{1cm} % Set the height of the header box
\setlength{\headsep}{0mm}     % Set the distance between the header box and the top of the text

\usepackage{gvv-book}
\usepackage{gvv}
\usepackage{cite}
\usepackage{amsmath,amssymb,amsfonts,amsthm}
\usepackage{algorithmic}
\usepackage{graphicx}
\usepackage{textcomp}
\usepackage{xcolor}
\usepackage{txfonts}
\usepackage{listings}
\usepackage{enumitem}
\usepackage{mathtools}
\usepackage{gensymb}
\usepackage{comment}
\usepackage[breaklinks=true]{hyperref}
\usepackage{tkz-euclide} 
\usepackage{listings}
\usepackage{gvv}                                        
\def\inputGnumericTable{}                                 
\usepackage[latin1]{inputenc}                                
\usepackage{color}                                            
\usepackage{array}  
\usepackage{longtable}                                       
\usepackage{calc}                                             
\usepackage{multirow}                                         
\usepackage{hhline}                                           
\usepackage{ifthen}                                           
\usepackage{lscape}
\usepackage{multicol}
\begin{document}

\bibliographystyle{IEEEtran}
\vspace{3cm}

\title{GATE-2016-CE}
\author{AI24BTECH11025-P.L.Kushal}
% \bigskip
{\let\newpage\relax\maketitle}

\renewcommand{\thefigure}{\theenumi}
\renewcommand{\thetable}{\theenumi}
\setlength{\intextsep}{10pt} % Space between text and floats


\numberwithin{equation}{enumi}
\numberwithin{figure}{enumi}
\renewcommand{\thetable}{\theenumi}
\begin{enumerate}[start=40]

\item The porosity  $n$  and the degree of saturation $S$ of a soil sample are 0.7 and $40\%$, respectively. In a $100 \, \text{m}^3$ volume of the soil, the volume (expressed in $m^3$) of air is \underline{\hspace{2cm}}.\\
\item A homogeneous gravity retaining wall supporting a cohesionless backfill is shown in the figure. The lateral active earth pressure at the bottom of the wall is 40 kPa. 


\begin{figure}[!ht]
\centering
\resizebox{0.4\textwidth}{!}{%
\begin{circuitikz}
\tikzstyle{every node}=[font=\large]
\draw  (6,9) rectangle (8.75,3.25);
\draw [short] (8.75,9) -- (13.75,9);
\draw [short] (12.25,8.25) -- (12.75,9);
\draw [short] (12.75,9) -- (13.25,8.25);
\draw [short] (12.75,8.25) -- (13.25,9);
\draw [short] (13.25,9) -- (13.75,8.25);
\draw [short] (9.25,3.25) -- (9.75,3.25);
\draw [short] (8.75,3) -- (8.75,2.5);
\draw [short] (6,3) -- (6,2.5);
\draw [<->, >=Stealth] (9.5,3.25) -- (9.5,9);
\draw [<->, >=Stealth] (6,2.75) -- (8.75,2.75);
\node [font=\normalsize] at (7,7.25) {Gravity};
\node [font=\normalsize] at (7,6.5) {Retaining};
\node [font=\normalsize] at (6.75,5.75) {wall};
\node [font=\normalsize] at (10,6.25) {6 m};
\node [font=\normalsize] at (7.25,2.5) {4 m};
\node [font=\normalsize] at (5.5,3.25) {P};
\node [font=\large] at (12.25,7) {Cohesionless};
\node [font=\large] at (12.25,6.25) {backfill};
\end{circuitikz}
}%

\label{fig:my_label}
\end{figure}





The minimum weight of the wall (expressed in kN per m length) required to prevent it from overturning about its toe (Point P) is  
\begin{multicols}{4}
\begin{enumerate}
\item 120 
\item 180
\item 240
\item 360
\end{enumerate} 
\end{multicols}




\item An undisturbed soil sample was taken from the middle of a clay layer (i.e., 1.5 m below GL), as shown in figure. The water table was at the top of clay layer. Laboratory test results are as follows:\\
\begin{flushleft}
    \begin{tabbing}
        \hspace*{3cm}\=\hspace*{4cm}\=\kill
        Natural water content of clay \> : \> 25\% \\
        Preconsolidation pressure of clay \> : \> 60 kPa \\
        Compression index of clay \> : \> 0.50 \\
        Recompression index of clay \> : \> 0.05 \\
        Specific gravity of clay \> : \> 2.70 \\
        Bulk unit weight of sand \> : \> 17 kN/m$^3$ \\
    \end{tabbing}
\end{flushleft}A compacted fill of 2.5 m height with unit weight of 20 kN/m$^3$ is placed at the ground level.
% diagram comes here

\begin{figure}[H]
\centering
\resizebox{0.5\textwidth}{!}{%
\begin{circuitikz}
\tikzstyle{every node}=[font=\large]
\draw [short] (5.75,7.5) -- (11.75,7.5);
\draw [short] (5.75,5.25) -- (11.75,5.25);
\draw [short] (5.75,2.75) -- (11.75,2.75);
\draw [short] (6,7.5) -- (6.25,7.25);
\draw [short] (6.25,7.5) -- (6,7.25);
\draw [short] (6.25,7.5) -- (6.5,7.25);
\draw [short] (6.25,7.25) -- (6.5,7.5);
\draw [short] (11.5,7.5) -- (11.25,7.25);
\draw [short] (11.25,7.5) -- (11.5,7.25);
\draw [short] (11.25,7.5) -- (11,7.25);
\draw [short] (11.25,7.25) -- (11,7.5);
\draw [short] (5,7.5) -- (5.5,7.5);
\draw [short] (5,5.25) -- (5.5,5.25);
\draw [short] (5,2.75) -- (5.5,2.75);
\draw [->, >=Stealth] (5.25,6.5) -- (5.25,7.5);
\draw [->, >=Stealth] (5.25,6) -- (5.25,5.25);
\draw [->, >=Stealth] (5.25,4.25) -- (5.25,5.25);
\draw [->, >=Stealth] (5.25,3.75) -- (5.25,2.75);
\draw [short] (10.5,5.5) -- (10.75,5.5);
\draw [short] (10.5,5.5) -- (10.75,5.25);
\draw [short] (10.75,5.25) -- (11,5.5);
\draw [short] (10.75,5.5) -- (11,5.5);
\draw [short] (10.5,5.25) -- (11,5.25);
\draw [short] (10.5,5.25) -- (11,5.25);
\draw [short] (10.5,5.25) -- (11,5.25);
\draw [short] (10.5,5) -- (11,5);
\draw [short] (10.75,5) -- (10.5,5);
\draw [short] (10.5,5.25) -- (11,5.25);
\node [font=\large] at (8.5,6.25) {Sand};
\node [font=\large] at (11.25,7.75) {GL};
\node [font=\large] at (12,6.25) {GWT};
\node [font=\large] at (9,4.25) {Clay};
\node [font=\large] at (9,2.25) {Hard stratum};
\node [font=\normalsize] at (5.25,4) {1 m};
\node [font=\normalsize] at (5.25,6.25) {1 m};
\end{circuitikz}
}%

\label{fig:my_label}
\end{figure}


 Assuming unit weight of water as 10 kN/m$^3$, the ultimate consolidation settlement (expressed in mm) of the clay layer is \underline{\hspace{3cm}}.\\
\item A seepage flow condition is shown in the figure. The saturated unit weight of the soil $\gamma_{\text{sat}} = 18 \, \text{kN/m}^3$. Using unit weight of water, $\gamma_w = 9.81 \, \text{kN/m}^3$, the effective vertical stress (expressed in kN/m$^2$) on plane X-X is \underline{\hspace{2cm}}.\\

\begin{figure}[!ht]
\centering
\resizebox{0.7\textwidth}{!}{%
\begin{circuitikz}
\tikzstyle{every node}=[font=\Huge]
\draw  (6.5,12.5) rectangle (15,-0.75);
\draw [short] (11.25,-0.75) -- (11.25,-7);
\draw [short] (10,-0.75) -- (10,-6);
\draw [short] (11.25,-7) -- (-6,-6.75);
\draw [short] (10,-5.75) -- (-4.75,-5.5);
\draw [short] (-4.75,-5.5) -- (-4.75,16.75);
\draw [short] (-6,-6.75) -- (-6,16.75);
\draw [short] (-4.75,16.75) -- (-1.5,16.75);
\draw [short] (-6,16.75) -- (-9,16.75);
\draw [short] (-9,16.75) -- (-9,18.75);
\draw [short] (-9,18.75) -- (-1.75,18.75);
\draw [short] (-1.5,16.75) -- (-1.5,18.75);
\draw [short] (9.5,12.25) -- (9.5,32);
\draw [short] (11.25,12.5) -- (11.25,32.25);
\draw [short] (-1.75,18.75) -- (-1.5,18.75);
\draw [short] (6.25,35.5) -- (15.75,35.5);
\draw [short] (15.75,35.5) -- (15.75,32.25);
\draw [short] (11.25,32.25) -- (15.75,32.25);
\draw [short] (9.5,32) -- (9.5,32.25);
\draw [short] (9.5,32.25) -- (5.5,32.25);
\draw [short] (5.5,32.25) -- (5.5,35.5);
\draw [short] (-1,18.5) -- (26,18.5);
\draw [dashed] (-9.5,1.5) -- (26.25,1.5);
\draw (5.5,35.5) to[short] (6.25,35.5);
\draw (14,35.5) to[short] (12.75,36.75);
\draw (12.75,36.75) to[short] (15,36.75);
\draw (14,35.5) to[short] (15.25,36.75);
\draw (15.25,36.75) to[short] (14,36.75);
\draw (13,35.25) to[short] (14.75,35.25);
\draw (13.5,35) to[short] (14.25,35);
\draw [short] (6,35.75) .. controls (5.75,36) and (6.25,36.75) .. (5,36.75);
\draw [short] (6.5,36.25) .. controls (6.25,36.5) and (6.75,37.25) .. (5.5,37.25);
\draw [short] (6.25,36) .. controls (6,36.25) and (6.5,37) .. (5.25,37);
\draw [short] (-11,16.75) .. controls (-10.75,17.25) and (-11.5,18.25) .. (-9.75,18.75);
\draw [short] (-10.75,16.5) .. controls (-10.5,17) and (-11.25,18) .. (-9.5,18.5);
\draw [short] (-11.25,17) .. controls (-11,17.5) and (-11.75,18.5) .. (-10,19);
\draw (-3,18.75) to[short] (-4.25,20);
\draw (-4.25,20) to[short] (-2,20);
\draw (-3,18.75) to[short] (-1.75,20);
\draw (-1.75,20) to[short] (-2.25,20);
\draw (-4,18.5) to[short] (-2.25,18.5);
\draw (-3.5,18.25) to[short] (-2.5,18.25);
\draw (22,12.75) to[short] (26.25,12.75);
\draw [short] (20.75,35.75) -- (25.25,35.75);
\draw [short] (23,35.75) -- (22,34.25);
\draw [short] (23,35.75) -- (24,34.5);
\draw [short] (23,35.75) -- (23,19);
\draw [short] (23,19.25) -- (23,18.75);
\draw [short] (23,18.5) -- (23,19);
\draw [short] (23,19) -- (23,19.25);
\draw [short] (23,18.5) -- (24.25,19.75);
\draw [short] (23,18.5) -- (21.75,20);
\draw [short] (23,35.75) -- (24.25,34.25);
\draw [short] (23,18.75) -- (24.5,20);
\draw [short] (23,18.5) -- (22.25,17.5);
\draw [short] (23.25,18.5) -- (24,17.5);
\draw [short] (23,18.5) -- (23.25,12.75);
\draw [short] (22.25,13.75) -- (23.25,12.75);
\draw [short] (23.25,12.75) -- (24.25,13.75);
\draw [short] (22,12.75) -- (21.25,12.75);
\draw [short] (23.25,12.75) -- (22.25,11.75);
\draw [short] (23.25,12.75) -- (24.25,11.75);
\draw [short] (23.25,12.5) -- (23.25,1.75);
\draw [short] (23.25,1.5) -- (23.25,2);
\draw [short] (23.25,1.5) -- (22.25,2.5);
\draw [short] (23.25,1.5) -- (24.25,2.5);
\draw [short] (21,-1) -- (25.75,-1);
\draw [short] (23.25,1.5) -- (23.25,-1);
\draw [short] (23.25,-1) -- (22.25,-0.25);
\draw [short] (23.25,-1) -- (24.25,-0.25);
\draw [short] (23.25,1.5) -- (22.25,0.75);
\draw [short] (23.25,1.5) -- (24.25,0.75);
\draw [short] (21.25,-6.75) -- (26,-6.75);
\draw [short] (23.25,-1) -- (22.5,-1.5);
\draw [short] (23.25,-1) -- (24,-1.5);
\draw [short] (23.25,-1) -- (23.25,-7);
\draw [short] (23.25,-6.75) -- (24,-6);
\draw [short] (23.25,-6.75) -- (22.5,-6);
\node [font=\Huge] at (-10.25,1.5) {X};
\node [font=\Huge] at (27,1.5) {X};
\node [font=\Huge] at (24.75,26.75) {3 m};
\node [font=\Huge] at (25.25,15.75) {1 m};
\node [font=\Huge] at (24.75,7.25) {5 m};
\node [font=\huge] at (24.75,0.25) {1 m};
\node [font=\huge] at (24.75,-4) {2 m};
\node [font=\Huge] at (10,7.75) {Soil};
\node [font=\Huge] at (10.5,6.25) {$\gamma_{sat} =  18kN/m^{3}$};
\end{circuitikz}
}%

\label{fig:my_label}
\end{figure}

\item A drained triaxial compression test on a saturated clay yielded the effective shear strength parameters as $c' = 15 \, \text{kPa}$ and $\phi' = 22^\circ$. Consolidated Undrained triaxial test on an identical sample of this clay at a cell pressure of 200 kPa developed a pore water pressure of 150 kPa at failure. The deviator stress (expressed in kPa) at failure is \underline{\hspace{2cm}}.\\

\item A concrete gravity dam section is shown in the figure. Assuming unit weight of water as 10 kN/m$^3$ and unit weight of concrete as 24 kN/m$^3$, the uplift force per unit length of the dam (expressed in kN/m) at PQ is \underline{\hspace{2cm}}.\\

\begin{figure}[!ht]
\centering
\resizebox{0.7\textwidth}{!}{%
\begin{circuitikz}
\tikzstyle{every node}=[font=\small]
\draw (2.25,8.25) to[short] (6.25,8.25);
\draw (6.25,8.25) to[short] (6.25,11.75);
\draw (6.25,11.25) to[short] (2.25,11.25);
\draw (6.25,11.75) to[short] (6.75,11.75);
\draw (6.75,11.75) to[short] (6.75,11);
\draw (6.75,11) to[short] (9.5,8.25);
\draw (6.25,8.25) to[short] (11.75,8.25);
\draw (8.75,9) to[short] (11,9);
\draw [->, >=Stealth] (9.75,10) -- (9.75,9);
\draw [->, >=Stealth] (10,7.5) -- (10,8.25);
\draw [<->, >=Stealth] (4,11.25) -- (4,8.25)node[pos=0.5, fill=white]{65 m};
\draw (9.5,7.25) to[short] (9.5,7.75);
\draw (6.75,7.25) to[short] (6.75,7.75);
\draw (6.25,7.25) to[short] (6.25,7.75);
\draw [->, >=Stealth] (5.5,7.5) -- (6.25,7.5);
\draw (6.75,8.75) to[short] (7.25,8.75);
\draw (6.75,8.75) to[short] (7.25,8.75);
\draw (6.75,8.75) to[short] (6.75,9);
\draw (7.25,8.75) to[short] (7,9);
\draw (6.75,8.75) to[short] (7.25,9.25);
\draw (7.25,8.75) to[short] (6.75,9.25);
\node [font=\small, color={rgb,255:red,231; green,13; blue,13}] at (6.25,8) {P};
\node [font=\small, color={rgb,255:red,231; green,13; blue,13}] at (9.5,8) {Q};
\node [font=\small, color={rgb,255:red,23; green,22; blue,22}] at (9.75,8.75) {5m};
\node [font=\small, color={rgb,255:red,23; green,22; blue,22}] at (6.5,7.5) {10m};
\draw [ color={rgb,255:red,23; green,22; blue,22}, <->, >=Stealth] (6.75,7.5) -- (9.5,7.5)node[pos=0.5, fill=white]{40m};
\draw [ color={rgb,255:red,23; green,22; blue,22}, dashed] (6.75,8.75) -- (6.75,8);
\draw [ color={rgb,255:red,23; green,22; blue,22}, dashed] (7.25,8.75) -- (7.25,8);
\draw [ color={rgb,255:red,23; green,22; blue,22}, ](2.75,8.25) to[short] (2.25,7.75);
\draw [ color={rgb,255:red,23; green,22; blue,22}, ](3,8.25) to[short] (2.5,7.75);
\draw [ color={rgb,255:red,23; green,22; blue,22}, ](2.75,8.25) to[short] (3.25,7.75);
\draw [ color={rgb,255:red,23; green,22; blue,22}, ](3,8.25) to[short] (3.5,7.75);
\draw [ color={rgb,255:red,23; green,22; blue,22}, ](5,8.25) to[short] (4.5,7.75);
\draw [ color={rgb,255:red,23; green,22; blue,22}, ](5,8.25) to[short] (5.5,7.75);
\draw [ color={rgb,255:red,23; green,22; blue,22}, ](4.25,7.75) to[short] (4.75,8.25);
\draw [ color={rgb,255:red,23; green,22; blue,22}, ](4.75,8.25) to[short] (5.25,7.75);
\draw [ color={rgb,255:red,23; green,22; blue,22}, ](10.5,7.75) to[short] (11,8.25);
\draw [ color={rgb,255:red,23; green,22; blue,22}, ](11,8.25) to[short] (11.5,7.75);
\draw [ color={rgb,255:red,23; green,22; blue,22}, ](10.75,7.75) to[short] (11.25,8.25);
\draw [ color={rgb,255:red,23; green,22; blue,22}, ](11.25,8.25) to[short] (11.75,7.75);
\node [font=\small, color={rgb,255:red,23; green,22; blue,22}] at (7,9.75) {Drain holes};
\draw [ color={rgb,255:red,23; green,22; blue,22}, ->, >=Stealth] (7.5,9.75) .. controls (8.25,9.25) and (8,8.75) .. (7.25,8.5) ;
\end{circuitikz}
}%
\label{fig:my_label}
\end{figure}

\item Seepage is occurring through a porous media shown in the figure. The hydraulic conductivity values $k_1, k_2, k_3$ are in m/day. The seepage discharge (m$^3$/day per m) through the porous media at section PQ is  

\begin{figure}[!ht]
\centering
\resizebox{1\textwidth}{!}{%
\begin{circuitikz}
\tikzstyle{every node}=[font=\normalsize]
\draw [short] (5.75,3.75) -- (16,3.75);
\draw [short] (6,3.75) -- (6,6.5);
\draw [short] (15.5,3.75) -- (15.5,7);
\draw [short] (6,6.5) -- (6.75,7.5);
\draw [short] (6.75,7.5) -- (8,8);
\draw [short] (8,8) -- (9.75,7.75);
\draw [short] (9.75,7.75) -- (11,8.5);
\draw [short] (11,8.5) -- (12.5,8.75);
\draw [short] (12.5,8.75) -- (13.5,7.75);
\draw [short] (13.5,7.75) -- (15.5,7);
\draw [short] (6,4.5) -- (15.5,4.5);
\draw [short] (8.25,4.5) -- (8.25,3.75);
\draw [short] (12.75,4.5) -- (12.75,3.75);
\draw [short] (10.5,4.5) -- (10.5,3.75);
\draw [short] (5.75,3.75) -- (4.75,3.75);
\draw [short] (16,3.75) -- (17.5,3.75);
\draw [short] (5,3.75) -- (5.25,3.5);
\draw [short] (5.25,3.75) -- (5.5,3.5);
\draw [short] (5.75,3.75) -- (6,3.5);
\draw [short] (5.5,3.75) -- (5.75,3.5);
\draw [short] (6,3.75) -- (6.25,3.5);
\draw [short] (6.25,3.75) -- (6.5,3.5);
\draw [short] (6.5,3.75) -- (6.75,3.5);
\draw [short] (6.75,3.75) -- (7,3.5);
\draw [short] (7,3.75) -- (7.25,3.5);
\draw [short] (7.25,3.75) -- (7.5,3.5);
\draw [short] (7.5,3.75) -- (7.75,3.5);
\draw [short] (8,3.75) -- (8.25,3.5);
\draw [short] (7.75,3.75) -- (8,3.5);
\draw [short] (8.25,3.75) -- (8.5,3.5);
\draw [short] (8.5,3.75) -- (8.75,3.5);
\draw [short] (8.75,3.75) -- (9,3.5);
\draw [short] (9,3.75) -- (9.25,3.5);
\draw [short] (9.25,3.75) -- (9.5,3.5);
\draw [short] (9.5,3.75) -- (9.75,3.5);
\draw [short] (9.75,3.75) -- (10,3.5);
\draw [short] (10,3.75) -- (10.25,3.5);
\draw [short] (10.25,3.75) -- (10.5,3.5);
\draw [short] (10.5,3.75) -- (10.75,3.5);
\draw [short] (10.75,3.75) -- (11,3.5);
\draw [short] (11,3.75) -- (11.25,3.5);
\draw [short] (11.25,3.75) -- (11.5,3.5);
\draw [short] (11.5,3.75) -- (11.75,3.5);
\draw [short] (11.75,3.75) -- (12,3.5);
\draw [short] (12,3.75) -- (12.25,3.5);
\draw [short] (12.25,3.75) -- (12.5,3.5);
\draw [short] (12.5,3.75) -- (12.75,3.5);
\draw [short] (12.75,3.75) -- (13,3.5);
\draw [short] (13,3.75) -- (13.25,3.5);
\draw [short] (13.25,3.75) -- (13.5,3.5);
\draw [short] (13.5,3.75) -- (15.5,3.75);
\draw [short] (13.5,3.75) -- (13.75,3.5);
\draw [short] (13.75,3.75) -- (14,3.5);
\draw [short] (14,3.75) -- (14.25,3.5);
\draw [short] (14.25,3.75) -- (14.5,3.5);
\draw [short] (14.5,3.75) -- (14.75,3.5);
\draw [short] (14.75,3.75) -- (15,3.5);
\draw [short] (15,3.75) -- (15.25,3.5);
\draw [short] (15.25,3.75) -- (15.5,3.5);
\draw [short] (15.5,3.75) -- (15.75,3.5);
\draw [short] (15.75,3.75) -- (16,3.5);
\draw [short] (16,3.75) -- (16.25,3.5);
\draw [short] (16.25,3.75) -- (16.5,3.5);
\draw [short] (16.5,3.75) -- (16.75,3.5);
\draw [short] (16.75,3.75) -- (17,3.5);
\draw [short] (17,3.75) -- (17.25,3.5);
\draw [short] (17.25,3.75) -- (17.5,3.5);
\draw [short] (4.75,3.75) -- (4.25,3.75);
\draw [short] (4.75,3.75) -- (5,3.5);
\draw [short] (4.5,3.75) -- (4.75,3.5);
\draw [short] (4.25,3.75) -- (4.5,3.5);
\draw [short] (4.25,3.75) -- (4,3.75);
\draw [short] (4,3.75) -- (4.25,3.5);
\draw [short] (4,3.75) -- (3.75,3.75);
\draw [short] (6,6.25) -- (4,6.25);
\draw [short] (5.25,4.5) -- (5.75,4.5);
\draw [<->, >=Stealth] (5.5,4.5) -- (5.5,3.75);
\draw [->, >=Stealth] (4.5,5.25) -- (4.5,6.25);
\draw [->, >=Stealth] (4.5,4.75) -- (4.5,3.75);
\draw [short] (15.75,4.5) -- (16.5,4.5);
\draw [short] (15.5,6.5) -- (17.25,6.5);
\draw [short] (6,3.25) -- (6,2.75);
\draw [short] (8.25,3.25) -- (8.25,2.75);
\draw [short] (10.5,3.25) -- (10.5,2.75);
\draw [short] (12.75,3.25) -- (12.75,2.75);
\draw [short] (15.5,3.25) -- (15.5,2.75);
\draw [->, >=Stealth] (17,5.25) -- (17,6.5);
\draw [->, >=Stealth] (17,4.75) -- (17,3.75);
\draw [->, >=Stealth] (6.75,3) -- (6,3);
\draw [->, >=Stealth] (7.25,3) -- (8.25,3);
\draw [->, >=Stealth] (9.25,3) -- (8.25,3);
\draw [->, >=Stealth] (9.75,3) -- (10.5,3);
\draw [->, >=Stealth] (11.25,3) -- (10.5,3);
\draw [->, >=Stealth] (11.75,3) -- (12.75,3);
\draw [->, >=Stealth] (13.75,3) -- (12.75,3);
\draw [->, >=Stealth] (14.25,3) -- (15.5,3);
\draw [short] (5.25,6.5) -- (5.5,6.5);
\draw [short] (5,6.5) -- (5.25,6.5);
\draw [short] (5,6.5) -- (5.25,6.25);
\draw [short] (5.5,6.5) -- (5.25,6.25);
\draw [short] (15.75,6.75) -- (16.25,6.75);
\draw [short] (15.75,6.75) -- (16,6.5);
\draw [short] (16.25,6.75) -- (16,6.5);
\node [font=\normalsize] at (4.5,5) {15 m};
\node [font=\normalsize] at (5,4.25) {3 m};
\node [font=\normalsize] at (10.25,7) {Impervious};
\node [font=\normalsize] at (7,4) {$k_{1}=2$};
\node [font=\normalsize] at (9.25,4) {$k_{2}=3$};
\node [font=\normalsize] at (10.5,4.75) {P};
\node [font=\normalsize] at (14,4) {$k_{3}=1$};
\node [font=\normalsize] at (10.5,3.25) {Q};
\node [font=\normalsize] at (7,3) {20 m};
\node [font=\normalsize] at (9.5,3) {10 m};
\node [font=\normalsize] at (11.5,3) {20 m};
\node [font=\normalsize] at (14,3) {10 m};
\node [font=\normalsize] at (16.5,4.25) {3 m};
\node [font=\normalsize] at (17,5) {10 m};
\draw [<->, >=Stealth] (16,3.75) -- (16,4.5);
\draw [short] (5,6) -- (5.5,6);
\draw [short] (15.75,6.25) -- (16.25,6.25);
\end{circuitikz}
}%

\label{fig:my_label}
\end{figure}


\begin{multicols}{4}
\begin{enumerate}
\item $\frac{7}{12}$
\item $\frac{1}{2}$
\item $\frac{9}{16}$
\item $\frac{3}{4}$
\end{enumerate} 
\end{multicols}

\item A 4 m wide rectangular channel, having bed slope of 0.001 carries a discharge of 16 m$^3$/s. Considering Manning's roughness coefficient = 0.012 and $g = 10 \, \text{m/s}^2$, the category of the channel slope is  (A) horizontal   (B) mild   (C) critical   (D) steep.
\begin{multicols}{4}
\begin{enumerate}
\item horizontal
\item mild
\item critical
\item steep
\end{enumerate} 
\end{multicols}
\item A sector gate is provided on a spillway as shown in the figure. Assuming $g = 10 \, \text{m/s}^2$, the resultant force per meter length (expressed in kN/m) on the gate will be \underline{\hspace{2cm}}.\\

\begin{figure}[!ht]
\centering
\resizebox{0.5\textwidth}{!}{%
\begin{circuitikz}
\tikzstyle{every node}=[font=\LARGE]
\draw (4.25,10.75) to[short] (4.25,10.75);
\draw (-4.5,9.75) to[short] (-4.5,-0.25);
\draw (-4.5,-0.25) to[short] (-0.25,-0.25);
\draw (-0.25,-0.25) to[short] (0.25,0.25);
\draw (0.25,0.25) to[short] (1,-0.5);
\draw (1,-0.5) to[short] (1.5,0);
\draw (1.5,0) to[short] (5,0);
\draw (-4.5,9.5) to[short] (-4.5,11.5);
\draw (5,0) to[short] (7.25,0);
\draw [short] (-4.5,11.5) .. controls (-1,15.25) and (-0.5,7.5) .. (3.25,3.5);
\draw [short] (3.25,3.5) .. controls (6,0) and (6.5,0.5) .. (8.75,0);
\draw [short] (7.25,0) -- (8.75,0);
\draw [short] (-3,12.5) -- (-3,21.75);
\draw [short] (-3,21.75) .. controls (-8.5,19.5) and (-7,13.75) .. (-3,12.5);
\draw [short] (-3,21.25) .. controls (-7.25,20) and (-6.75,14.5) .. (-3,13);
\draw [short] (-3,21.75) -- (5.75,16.75);
\draw [short] (-3,12.5) -- (5.75,16.75);
\draw [short] (5.75,16.75) -- (-6,17);
\draw [short] (-3,21.75) -- (-11.25,21.75);
\draw [short] (-9.75,22) -- (-9.25,21.75);
\draw [short] (-9.25,21.75) -- (-8.75,22);
\draw [short] (-8.75,22) -- (-9.75,22);
\draw [short] (-9.75,21.5) -- (-8.75,21.5);
\draw [short] (-9.5,21.25) -- (-9,21.25);
\draw [->, >=Stealth] (-7,20.25) -- (-5.5,20.25);
\draw [<->, >=Stealth] (-2.5,22.75) -- (6.25,17.5);
\draw [short] (-2.75,22.25) -- (-2.25,23.25);
\draw [short] (6,17) -- (6.5,17.75);
\node [font=\Huge] at (-1.25,5) {Spillway};
\node [font=\Huge] at (-9.5,20.25) {Sector gate};
\node [font=\huge] at (2.25,21) {5 m};
\draw [short] (4,16.75) .. controls (4,17.5) and (4,17.5) .. (4.5,17.5);
\draw [short] (4,16.75) .. controls (4,16.25) and (4.25,16.25) .. (4.75,16.25);
\node [font=\LARGE] at (3,17.25) {$30^{o}$};
\node [font=\LARGE] at (9.5,16.75) {};
\node [font=\LARGE] at (3,16) {$30^{o}$};
\end{circuitikz}
}%

\label{fig:my_label}
\end{figure}

\item A hydraulically efficient trapezoidal channel section has a uniform flow depth of 2 m. The bed width (expressed in m)\\

\item Effluent from an industry 'A' has a pH of 4.2. The effluent from another industry 'B' has double the hydroxyl (OH) ion concentration than the effluent from industry 'A'. pH of effluent from the industry 'B' will be \underline{\hspace{3cm}}.\\

\item An electrostatic precipitator (ESP) with $5600 m^{2}$ of collector plate area is 96 percent efficient in treating 185 m$^{3}$/s of flue gas from a 200 MW thermal power plant. It was found that in order to achieve 97 percent efficiency, the collector plate area should be 6100 m$^{2}$. In order to increase the efficiency to 99 percent, the ESP collector plate area (expressed in m$^{2}$) would be \underline{\hspace{3cm}}.\\

\item The 2-day and 4-day BOD values of a sewage sample are 100 mg/L and 155 mg/L, respectively. The value of BOD rate constant (expressed in per day) is \underline{\hspace{3cm}}. \\

\end{enumerate}
\end{document}