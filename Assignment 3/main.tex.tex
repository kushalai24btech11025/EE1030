\let\negmedspace\undefined
\let\negthickspace\undefined
\documentclass[journal]{IEEEtran}
\usepackage[a5paper, margin=10mm, onecolumn]{geometry}
%\usepackage{lmodern} % Ensure lmodern is loaded for pdflatex
\usepackage{tfrupee} % Include tfrupee package

\setlength{\headheight}{1cm} % Set the height of the header box
\setlength{\headsep}{0mm}     % Set the distance between the header box and the top of the text

\usepackage{gvv-book}
\usepackage{gvv}
\usepackage{cite}
\usepackage{amsmath,amssymb,amsfonts,amsthm}
\usepackage{algorithmic}
\usepackage{graphicx}
\usepackage{textcomp}
\usepackage{xcolor}
\usepackage{txfonts}
\usepackage{listings}
\usepackage{enumitem}
\usepackage{mathtools}
\usepackage{gensymb}
\usepackage{comment}
\usepackage[breaklinks=true]{hyperref}
\usepackage{tkz-euclide} 
\usepackage{listings}
\usepackage{amsmath}
% \usepackage{gvv}                                        
\def\inputGnumericTable{}                                 
\usepackage[latin1]{inputenc}                                
\usepackage{color}                                            
\usepackage{array}                                            
\usepackage{longtable}                                       
\usepackage{calc}  
\usepackage{circuitikz}
\usepackage{multirow}                                         
\usepackage{hhline}                                           
\usepackage{ifthen}                                           
\usepackage{lscape}
\usepackage{tikz}
\begin{document}

\bibliographystyle{IEEEtran}
\vspace{3cm}


\renewcommand{\thefigure}{\theenumi}
\renewcommand{\thetable}{\theenumi}
\setlength{\intextsep}{10pt} % Space between text and floats


\numberwithin{equation}{enumi}
\numberwithin{figure}{enumi}
\renewcommand{\thetable}{\theenumi}

\title{GATE ME-2008}
\author{AI24BTECH11025 PEDAPROLU LAKSHMI KUSHAL}
\maketitle
\renewcommand{\thefigure}{\theenumi}
\renewcommand{\thetable}{\theenumi}
\begin{enumerate}[start=35]
    \item The natural frequency of the spring mass system shown in figure is clossest to

    \begin{figure}[!ht]
\centering
\resizebox{0.4\textwidth}{!}{%
\begin{circuitikz}
\tikzstyle{every node}=[font=\normalsize]
\draw (2,11.75) to[short] (2,9.5);
\draw (8.25,11.75) to[short] (8.25,9.75);
\draw  (4.25,11.25) rectangle (6,10);
\draw (2,10.75) to[R] (4.25,10.75);
\draw (6,10.75) to[R] (8.25,10.75);
\draw (2,11.75) to[short] (1.5,11.25);
\draw (2,11.25) to[short] (1.5,10.75);
\draw (2,10.75) to[short] (1.5,10.25);
\draw (2,10.25) to[short] (1.5,9.75);
\draw (2,9.75) to[short] (1.5,9.25);
\draw (8.25,11.75) to[short] (8.75,11.25);
\draw (8.25,11.25) to[short] (8.75,10.75);
\draw (8.25,10.75) to[short] (8.75,10.25);
\draw (8.25,10.25) to[short] (8.75,9.75);
\draw (8.25,9.75) to[short] (8.75,9.25);
\node [font=\normalsize] at (7,10) {K2=1600N/m};
\node [font=\normalsize] at (3,10.25) {K1=4000N/m};
\node [font=\normalsize] at (5,10.75) {m=1.4kg};
\end{circuitikz}
}%

\label{fig:my_label}
\end{figure}
    
    \begin{multicols}{4}
    \begin{enumerate}
        \item $8$ Hz
        \item $10$ Hz
        \item $12$ Hz
        \item $14$ Hz
    \end{enumerate}
    \end{multicols}
    \item The rod PQ of length L and with flexural rigidity EI is hinged at both ends. For what minimum force is it expected to buckle?

    \begin{figure}[!ht]
\centering
\resizebox{0.4\textwidth}{!}{%
\begin{circuitikz}
\tikzstyle{every node}=[font=\large]
\draw (-2.75,12.75) to[short] (-2.75,5);
\draw (-2.75,5) to[short] (4.5,5);
\draw [short] (-2.75,11.5) -- (-1.75,10.5);
\draw [short] (-1.75,10.5) -- (-2.75,9.75);
\draw [short] (-1.75,10.5) -- (2.25,6.25);
\draw  (2,5.25) circle (0.25cm);
\draw  (2.75,5.25) circle (0.25cm);
\draw [short] (1.5,5.5) -- (3,5.5);
\draw [short] (2.25,6.25) -- (1.5,5.5);
\draw [short] (2.25,6.25) -- (3,5.5);
\draw [dashed] (2.25,6.25) -- (-0.25,6.25);
\draw [short] (1.5,7) .. controls (1,7) and (1,6.75) .. (1,6.25);
\node [font=\large] at (0.5,6.75) {45};
\node [font=\small] at (0.75,7) {o};
\draw [->, >=Stealth] (4.5,6.25) -- (2.5,6.25);
\node [font=\large] at (4.75,6.25) {\textbf{F}};
\node [font=\large] at (-1.5,11) {\textbf{p}};
\node [font=\large] at (2.25,7) {\textbf{Q}};
\draw [short] (-2.75,12.75) -- (-3.25,12);
\draw [short] (-2.75,12) -- (-3.25,11.25);
\draw [short] (-2.75,11) -- (-3.25,10.5);
\draw [short] (-2.75,10.25) -- (-3.25,9.75);
\draw [short] (-2.75,9.25) -- (-3.25,8.75);
\draw [short] (-2.75,8.5) -- (-3.25,8);
\draw [short] (-2.75,7.75) -- (-3.25,7.25);
\draw [short] (-2.75,7) -- (-3.25,6.5);
\draw [short] (-2.75,6) -- (-3.25,5.5);
\draw [short] (-2.75,5.25) -- (-3.25,4.75);
\draw [short] (-2.25,5) -- (-2,4.5);
\draw [short] (-1.75,5) -- (-1.5,4.5);
\draw [short] (-2.75,5) -- (-2.5,4.5);
\draw [short] (-1.25,5) -- (-1,4.5);
\draw [short] (-0.75,5) -- (-0.5,4.5);
\draw [short] (-0.25,5) -- (0,4.5);
\draw [short] (0.25,5) -- (0.5,4.5);
\draw [short] (0.75,5) -- (1,4.5);
\draw [short] (1.25,5) -- (1.5,4.5);
\draw [short] (2,5) -- (2.25,4.5);
\draw [short] (2.5,5) -- (2.75,4.5);
\draw [short] (3.25,5) -- (5.25,5);
\draw [short] (3,5) -- (3.25,4.5);
\draw [short] (3.5,5) -- (3.75,4.5);
\draw [short] (3.75,5) -- (4,4.5);
\draw [short] (4.25,5) -- (4.5,4.5);
\end{circuitikz}
}%

\label{fig:my_label}
\end{figure}
    
    \begin{multicols}{2}
    \begin{enumerate}
        \item $\pi^2 EI/L^2$
        \item $\sqrt{2}\pi^2 EI/L^2$
        \item $\pi^2 EI/\sqrt{2}L^2$
        \item $\pi^2 EI/2L^2$
    \end{enumerate}
        \end{multicols}
    
    \item In  a cam design, the rise motion is given by a simple harmonic motion (SHM) \\$s=(h/2)(1-\cos{\pi\theta/\beta})$ where h is total rise, $\theta$ is camshaft angle, B is the total angle of the rise interval. The jerk is given by
    \begin{multicols}{2}
    \begin{enumerate}
        \item $h/2(1-\cos{\pi\theta/\beta})$
        \item $\pi/\beta h/2 \sin{\pi\theta/\beta}$
        \item $\pi^2/\beta^2 h/2 \cos{\pi\theta/\beta}$
        \item $-\pi^3/\beta^3 h/2 \cos{\pi\theta/\beta}$
    \end{enumerate}
    \end{multicols}
    \item A uniform rigid rod of mass $m=1kg$ and length $l= 1m$ is hinged at center and laterally supported at one end by a spring of spring constant $k = 300N/m$. The natural frequency $w_n$ in rad/s is 
    \begin{multicols}{4}
    \begin{enumerate}
        \item $10$
        \item $20$
        \item $30$
        \item $40$
    \end{enumerate}
    \end{multicols}
    \item A comprension spring is made of music wire of $2 mm$ diameter having a shear strength and shear modulus of $800 MPa$ and $80$ GPa respectively. The mean coil diameter is $20$ mm, free length is $40$ mm and the number of active coils is $10$. If the mean coil diameter is reduced to $10$ mm, the stiffness of the spring is approximately
    \begin{multicols}{4}
    \begin{enumerate}
        \item decreased by $8$ times
        \item decreased by $2$ times
        \item increased by $2$ times.
        \item increased by $8$ times.
    \end{enumerate}
    \end{multicols}

    \item A journal bearing has a shaft diameter of $40$ mm and a length of $40$ mm. The shaft is rotating at $20$ rad's and the viscosity of the lubricant is $20$ mPas. The clearance is $0.020$ mm. The loss of torque due to the viscosity of the lubricant is approximately
    \begin{multicols}{4}
    \begin{enumerate}
        \item $0.040$ Nm
        \item $0.252$ Nm
        \item $0.400$ Nm
        \item $0.652$ Nm.
    \end{enumerate}
    \end{multicols}

    \item A clutch has outer and inner diameters $100$ mm and $40$ mm respectively. Assuming a uniform pressure of $2$ MPa and coefficient of friction of liner material $0.4$, the torque carrying capacity of the church is
    \begin{multicols}{4}
    \begin{enumerate}
        \item $148$ Nm
        \item $196$ Nm
        \item $372$ Nm
        \item $490$ Nm 
    \end{enumerate}
    \end{multicols}

    \item A spur gear has a module of $3$ mm, number of teeth $16$, a face width of $36$ mm and a pressure angle of $20$. It is transmitting a power of $3$ kW at $20$ rev/s. Taking a velocity factor of $1.5$, and a form facer of $0.3$, the stress in the gear tooth is about
    \begin{multicols}{4}
    \begin{enumerate}
        \item $32$ MPa
        \item $46$ MPa
        \item $58$ MPa
        \item $70$ MPa
    \end{enumerate}
    \end{multicols}

    \item Match the type of gears with their most appropriate description.\\
    \begin{tabular}{|c|c|p{1cm}|}
    \hline
    \textbf{Type of gear} & \textbf{Description} \\
    \hline
    P: Helical & 1. Axes non-parallel and non-intersecting \\
    \hline
    Q: Spiral Bevel & 2. Axes parallel and teeth are inclined to the axis \\
    \hline
    R: Hypoid & 3. Axes parallel and teeth are parallel to the axis \\
    \hline
    S: Rack and pinion & 4. Axes are perpendicular and intersecting, and teeth are inclined to the axis \\
    \hline
    & 5. Axes are perpendicular and used for large speed reduction \\
    \hline
    & 6. Axes parallel and one of the gears has infinite radius \\
    \hline 
    \end{tabular}
    
    \begin{multicols}{2}
    \begin{enumerate}
        \item P-$2$, Q-$4$, R-$1$, S-$6$
        \item P-$1$, Q-$4$, R-$5$, S-$6$
        \item P-$2$, Q-$6$, R-$4$, S-$2$
        \item P-$6$, Q-$3$, R-$1$, S-$5$
    \end{enumerate}
    \end{multicols}

    \item A gas expands in a frictionless piston-cylinder arrangement. The expansion process is very slow, and is resisted by an ambient pressure of $100$ kPa. During the expansion process, the pressure of the system (gas) remains constant at $300$ kPa. The change in volume of the gas in $0.01$ $m^2$. The maximum amount of work that that could be utlized from the above process is
    \begin{multicols}{4}
    \begin{enumerate}
        \item $0$ KJ
        \item $1$ kJ
        \item $2$ KJ
        \item $3$ KJ
    \end{enumerate}
    \end{multicols}

    \item The logarithmic mean temperature difference (LMTD) of a counterflow heat exchanger is $20^o C$. The cold fluid enters at $20^o C$ and the hot fluid enters at $100^o C$. Mass flow rate of the cold fluid is twice that of the hot fluid. Specific heat at constant pressure of the hot fluid is twice that of the cold fluid. The exit temperature of the colt fluid
    \begin{multicols}{2}
    \begin{enumerate}
        \item is $40^o C$
        \item is $60^o C$
        \item is $30^o C$
        \item cannot be determined
    \end{enumerate}
    \end{multicols}
    
    \item A two dimensional fluid element rotates like a rigid body. At a point within the element, the pressure is $1$ unit. Radius of the Mohr's circle, characterizing the state of stress at that point, is
    \begin{multicols}{4}
    \begin{enumerate}
        \item $0.5$ unit
        \item 0 unit
        \item 1 unit
        \item 2 units
    \end{enumerate}
    \end{multicols}
    \item A cyclic device operates between three thermal reservoirs, as shown in the figure Heat is transferred to/from the cyclic device. It is assumed that heat transfer between each thermal reservoir and the cyclic device takes place across negligible temperature difference. Interactions between the cyclic device and the respective thermal reservoirs that are shown in the figure are all in the form of beat transfer.

    \begin{figure}[!ht]
\centering
\resizebox{0.4\textwidth}{!}{%
\begin{circuitikz}
\tikzstyle{every node}=[font=\Huge]
\draw  (-9.75,10.75) rectangle (-6.5,8);
\draw  (-5.75,10.75) rectangle (-2.25,8);
\draw  (-1.25,10.75) rectangle (2,8);
\draw  (-4,2) circle (2.75cm);
\draw [->, >=Stealth] (-4.25,8) -- (-4.25,4.5);
\draw [short] (-9.25,8) -- (-9.25,1.5);
\draw [short] (1.25,8) -- (1.25,1.5);
\draw [->, >=Stealth] (-9.25,1.5) -- (-6.75,1.5);
\draw [->, >=Stealth] (1.25,1.5) -- (-1.25,1.5);
\node [font=\huge] at (-8.25,9.25) {1000 K};
\node [font=\huge] at (-4.25,9.25) {500 K};
\node [font=\huge] at (0.25,9.25) {300 K};
\node [font=\LARGE] at (-8,2.75) {100 KJ};
\node [font=\LARGE] at (-3.25,6) {50 KJ};
\node [font=\LARGE] at (-0.25,2.5) {60 KJ};
\node [font=\Huge] at (-4,2.5) {Cyclic};
\node [font=\Huge] at (-4,1.5) {device};
\end{circuitikz}
}%

\label{fig:my_label}
\end{figure}
    
    The cyclic device can be
    \begin{enumerate}
        \item  reversible heat engine
        \item a reversible heat pump or a reversible refrigerator
        \item an irreversible beat engine
        \item n irreversible heat pump or as irreversible refrigerator
    \end{enumerate}
    \item A balloon containing an ideal gas is initially kept in an evacuated and insulated room. The balloon ruptures and the gas fills up the entire room. Which one of the following statements is TRUE at the end of above process?
    \begin{enumerate}
        \item The internal energy of the gas decreases from it's initial value, but the enthalpy remains constant
        \item The internal energy of the gas increases from its initial value, but the enthalpy remains constant 
        \item Both internal energy and enthalpy of the gas remain constant
        \item Both internal energy and enthalpy of the gas increase
    \end{enumerate}

    \item A rigid, insulated tank is initially evacuated. The tank is connected with a supply line through which air (assumed to be ideal gas with constant specific heats) passes at $1$ MPa, $350^o C$. A valve connected with the supply line is opened and the tank is charged with air until the final pressure inside the tank reaches $1$ MPa. The final temperature inside the tank

    \begin{figure}[!ht]
\centering
\resizebox{0.3\textwidth}{!}{%
\begin{circuitikz}
\tikzstyle{every node}=[font=\Large]
\draw [->, >=Stealth] (-6.75,12.25) -- (-1.25,12.25);
\draw [->, >=Stealth] (-4,12.25) -- (-4,10.5);
\draw  (-4,9.75) circle (0.75cm);
\draw [->, >=Stealth] (-4,9) -- (-4,6.75);
\draw  (-5.25,6.75) rectangle (-2.75,4.25);
\draw [short] (-4.5,10.25) -- (-3.5,9.25);
\draw [short] (-3.5,10.25) -- (-4.5,9.25);
\node [font=\Large] at (-4,13.5) {\textbf{Air supply}};
\node [font=\Large] at (-4,12.75) {\textbf{line}};
\node [font=\Large] at (-2.25,9.75) {\textbf{valve}};
\node [font=\Large] at (-4,5.5) {\textbf{Tank}};
\end{circuitikz}
}%

\label{fig:my_label}
\end{figure}
    
    \begin{enumerate}
        \item is greater than $350^o C$
        \item is less than $350^o C$
        \item is equal to $350^o C$
        \item may be greater than, less than, or equal to $350^o C$, depending on the volume of the tank
    \end{enumerate}
    \item For the three-dimensional object shown in the figure below, five faces are insulated. The sixth face (PQRS), which is not insulated, interacts thermally with the ambient, with a convective heat transfer coefficient of $10W/m^2 .K$. The ambient temperature is $30^o C$. Heat is uniformly generated inside the object at the rate of $100W/m^3$. Assuming the face PQRS to be at uniform temperature, its steady state temperature is\\\\ 
    
    \begin{figure}[!ht]
\centering
\resizebox{0.4\textwidth}{!}{%
\begin{circuitikz}
\tikzstyle{every node}=[font=\Large]
\draw  (-7.25,10.25) rectangle (-5,5.75);
\draw [short] (-7.25,10.25) -- (-3.75,12);
\draw [short] (-5,10.25) -- (-1.75,12);
\draw [short] (-1.75,12) -- (-1.75,7.75);
\draw [short] (-3.75,12) -- (-1.75,12);
\draw [short] (-5,5.75) -- (-1.75,7.75);
\draw [dashed] (-3.75,12) -- (-3.75,7.75);
\draw [dashed] (-1.75,7.75) -- (-3.75,7.75);
\draw [dashed] (-7.25,5.75) -- (-3.75,7.75);
\draw [short] (-8.25,10.25) -- (-7.5,10.25);
\draw [short] (-8.25,5.75) -- (-7.5,5.75);
\draw [short] (-7.25,5.5) -- (-7.25,4.75);
\draw [short] (-5,5.5) -- (-5,4.75);
\draw [short] (-4.75,5.75) -- (-3.75,5.75);
\draw [short] (-1.5,7.75) -- (-0.5,7.75);
\draw [<->, >=Stealth] (-7.75,10.25) -- (-7.75,5.75);
\draw [<->, >=Stealth] (-7.25,5.25) -- (-5,5.25);
\draw [<->, >=Stealth] (-4.25,5.75) -- (-1,7.75);
\node [font=\Large] at (-8.25,8) {\textbf{2m}};
\node [font=\Large] at (-2,6.25) {\textbf{2m}};
\node [font=\Large] at (-6.25,4.75) {\textbf{1m}};
\node [font=\Large] at (-7.5,10.5) {\textbf{P}};
\node [font=\Large] at (-3.75,12.5) {\textbf{Q}};
\node [font=\Large] at (-4,8) {\textbf{R}};
\node [font=\Large] at (-7.75,5.25) {\textbf{S}};
\node [font=\Large] at (-4.75,10.75) {\textbf{E}};
\node [font=\Large] at (-1.5,12.25) {\textbf{F}};
\node [font=\Large] at (-1.25,8) {\textbf{G}};
\node [font=\Large] at (-4.5,5) {\textbf{H}};
\end{circuitikz}
}%

\label{fig:my_label}
\end{figure}
    
    \begin{multicols}{4}
        \begin{enumerate}
            \item $10^o C$
            \item $20^o C$
            \item $30^o C$
            \item $40^o C$
        \end{enumerate}
    \end{multicols}

    \item Water, having a density of $1000 kg/m^3$, issues from a nozzle with a velocity of $10 m/s$ and the jet strikes a bucket mounted on a Pelton wheel. The wheel rotates at $10 rad/s$. The mean diameter of the wheel is $1 m$. The jet is split into two equal streams by the bucket, such that each stream is deflected by $120^o$, as shown in the figure, Friction in the bucket may be neglected. Magnitude of the torque exerted by the water on the wheel, per unit mass flow rate of the incoming jet, is

    \begin{figure}[!ht]
\centering
\resizebox{0.4\textwidth}{!}{%
\begin{circuitikz}
\tikzstyle{every node}=[font=\large]
\draw [line width=1pt, ->, >=Stealth] (-8,8.25) -- (-4,8.25);
\draw [line width=1pt, ->, >=Stealth] (-4,8.25) -- (-5.25,10.75);
\draw [line width=1pt, ->, >=Stealth] (-4,8.25) -- (-5,6);
\draw [line width=0.6pt, dashed] (-4,8.25) -- (-0.5,8.25);
\draw [line width=0.6pt, short] (-4.5,9.25) .. controls (-3.5,9.25) and (-3.25,9) .. (-3,8.25);
\draw [line width=0.6pt, short] (-4.5,7.25) .. controls (-3.5,7.25) and (-3.25,7.25) .. (-3,8.25);
\node [font=\Large] at (-2.75,9.5) {120};
\node [font=\Large] at (-3,6.75) {120};
\node [font=\normalsize] at (-2.5,7) {0};
\node [font=\normalsize] at (-2.25,9.75) {0};
\node [font=\large] at (-6.5,8.5) {Incoming jet};
\node [font=\large] at (-3.75,10.75) {Deflected jet};
\node [font=\large] at (-3.5,5.75) {Deflected jet};
\end{circuitikz}
}%

\label{fig:my_label}
\end{figure}
    
    \begin{multicols}{2}
        \begin{enumerate}
            \item $0 (N.m)/(kg/s)$
            \item $1.25(N.m)/(kg/s)$
            \item $2.5(N.m)/(kg/s)$
            \item $3.75(N.m)/(kg/s)$
        \end{enumerate}
    \end{multicols}



    


\end{enumerate}

\end{document}
