\let\negmedspace\undefined
\let\negthickspace\undefined
\documentclass[journal]{IEEEtran}
\usepackage[a5paper, margin=10mm, onecolumn]{geometry}
%\usepackage{lmodern} % Ensure lmodern is loaded for pdflatex
\usepackage{tfrupee} % Include tfrupee package

\setlength{\headheight}{1cm} % Set the height of the header box
\setlength{\headsep}{0mm}     % Set the distance between the header box and the top of the text

\usepackage{gvv-book}
\usepackage{gvv}
\usepackage{cite}
\usepackage{amsmath,amssymb,amsfonts,amsthm}
\usepackage{algorithmic}
\usepackage{graphicx}
\usepackage{textcomp}
\usepackage{xcolor}
\usepackage{txfonts}
\usepackage{listings}
\usepackage{enumitem}
\usepackage{mathtools}
\usepackage{gensymb}
\usepackage{comment}
\usepackage[breaklinks=true]{hyperref}
\usepackage{tkz-euclide} 
\usepackage{listings}
% \usepackage{gvv}                                        
\def\inputGnumericTable{}                                 
\usepackage[latin1]{inputenc}                                
\usepackage{color}                                            
\usepackage{array}                                            
\usepackage{longtable}                                       
\usepackage{calc}                                             
\usepackage{multirow}                                         
\usepackage{hhline}                                           
\usepackage{ifthen}                                           
\usepackage{lscape}
\begin{document}

\bibliographystyle{IEEEtran}
\vspace{3cm}

\title{1 - 1.4 - 3}
\author{AI24BTECH11025 - PEDAPROLU LAKSHMI KUSHAL
}
% \maketitle
% \newpage
% \bigskip
{\let\newpage\relax\maketitle}

\renewcommand{\thefigure}{\theenumi}
\renewcommand{\thetable}{\theenumi}
\setlength{\intextsep}{10pt} % Space between text and floats


\numberwithin{equation}{enumi}
\numberwithin{figure}{enumi}
\renewcommand{\thetable}{\theenumi}


\textbf{Question}:\\

\textbf{Find the ratio in which the point \(P \) $\brak{\frac{3}{4}, \frac{5}{12} }$ divides the line segment joining the points \(A \) $\brak{\frac{1}{2},\frac{3}{2}}$ and \(B \) $\brak{2,-5}$.}
\\

\textbf{Solution:}\\
To find the ratio \( m : n \) in which the point \( P\left(\frac{3}{4}, \frac{5}{12}\right) \) divides the line segment joining the points \( A\left(\frac{1}{2}, \frac{3}{2}\right) \) and \( B(2, -5) \), we use the section formula.

Let the ratio be \( m : n \).

The coordinates of the point \( P \) dividing the line segment joining \( A \) and \( B \) are given by:

\[
\left( \frac{m \cdot x_2 + n \cdot x_1}{m+n}, \frac{m \cdot y_2 + n \cdot y_1}{m+n} \right)
\]

Substituting the given coordinates:

\[
\left( \frac{m \cdot 2 + n \cdot \frac{1}{2}}{m+n}, \frac{m \cdot (-5) + n \cdot \frac{3}{2}}{m+n} \right) = \left(\frac{3}{4}, \frac{5}{12}\right)
\]

Equating the corresponding coordinates:

\[
\frac{m \cdot 2 + n \cdot \frac{1}{2}}{m+n} = \frac{3}{4}
\] let this be equation 1.

\[
\frac{m \cdot (-5) + n \cdot \frac{3}{2}}{m+n} = \frac{5}{12}
\] and let this equation be equation 2.

The first equation can be rearranged as $ m\cdot8 + n\cdot2 = m\cdot3 + n\cdot3 $ which upon further simplification gives 5m = n.\\
Even the second equation gives the same result.\\
Therefore the ratio m : n in which the point \(P \) divides the line segment joining \(A\) and \(B\) in 1 : 5.\\
Hence , the answer is \textbf{1:5} .
\end{document}